\documentclass[a4paper, openany]{report}
\usepackage[12pt]{moresize}
\usepackage{pgfkeys}
\usepackage{enumitem}

\usepackage{bookmark}
\usepackage{multicol}
\usepackage{geometry}
\geometry{ a4paper, total={170mm,257mm}, left=10mm, right=10mm, top=1mm, bottom=20mm, }
\usepackage{epigraph}


%%%%%
% Race
%%%%%

% Set up the keys.  Only the ones directly under /myparbox
% can be accepted as options to the \myparbox macro.
\pgfkeys{
    /race/.is family, /race,
    % Here are the options that a user can pass
    default/.style = {
        name = Unnamed,
        brawn = 2,
        agility = 2,
        intellect = 2,
        cunning = 2,
        willpower = 2,
        presence = 2,
        wounds = 10 + Brawn,
        strain = 10 + Willpower,
        xp = 10 XP,
    },
    name/.estore in = \raceName,
    brawn/.estore in = \raceBrawn,
    agility/.estore in = \raceAgility,
    intellect/.estore in = \raceIntellect,
    cunning/.estore in = \raceCunning,
    willpower/.estore in = \raceWillpower,
    presence/.estore in = \racePresence,
    wounds/.estore in = \raceWounds,
    strain/.estore in = \raceStrain,
    xp/.estore in = \raceXP,
}

\newcommand\Race[9][]{
    \pgfkeys{/race, default, #1}%

    \vspace{1em}
    \belowpdfbookmark{\raceName}{race-\raceName}
    \noindent{\LARGE\sffamily \raceName}\\

    \noindent\begin{minipage}[c]{\linewidth} #2
    \end{minipage}
    \vspace{.2em}

    \begin{itemize}[topsep=0pt, partopsep=0pt]
        \setlength{\itemsep}{0pt}%
        \item \textbf{Wound Threshold:} \raceWounds
        \item \textbf{Strain Threshold:} \raceStrain
        \item \textbf{Starting Experience:} \raceXP
        \item \textbf{Special Ability:} #3
        #4
        #5
        #6
        #7
        #8
        #9
    \end{itemize}
}

\begin{document}

\chapter{Races}
\setcounter{unbalance}{25}
\begin{multicols*}{2}

\Race[name=Dwarf, presence=1]
{
\epigraph{\textit{"The worst thing you can say to a dwarf is 'It can’t be done.'
If he’s already decided to do it, he may never speak to you again. If he hasn’t
decided to take up the task, he may commit himself to it simply out of spite.
‘Impossible’ is not a concept most dwarves understand. Anything can be done,
with enough determination."} }{Sha’len, Nibenese trader}

Dwarves form a good part of the people encountered in the Tablelands. These
strong and devoted beings live to fulfill their focus, a task they choose
to devote their lives to. Stubborn and strong‐minded, dwarves make good
companions, even though their usual focused nature can tend to be bothersome.\\

\textbf{Personality:} Dwarves prefer to occupy themselves with meaningful tasks,
and often approach these tasks with an intensity rarely seen in other races.
As such, dwarves make excellent laborers, and take great pride in their
accomplishments. However, their stubbornness can lead to difficulties.
Dwarves will sometimes fail to listen to reason, attempting to accomplish what
are impossible tasks. Dwarves live for their focus. Dwarves that die while
being unable to complete their focus return from the dead as banshees to
haunt their unfinished work. A dwarf also rarely divulges his focus to anyone.\\
}
{Special Ability with } {\item \textbf{Named Option: } blaat} {} {} {} {} {}

\Race[name=Elves, presence=1]
{
\epigraph{\textit{"Honor? The word does not exist in the Elven language."} }
{Tharak, human guard}

Athasʹ deserts, plains, steppes and badlands are home
to the elves, a long–limbed race of trading, raiding,
thieving sprinters. Running is the key to acceptance and
respect among elves. Elves that are injured and cannot
run are often left behind to die.\\

\textbf{Personality:} Other races see elves as dishonest and
lazy; generally a fair assessment. Elves idle around their
time for days until compelled by need to exert themselves,
but they can run for days without complaint. No self–
respecting elf will consent to ride an animal. To do so is
dishonorable; Elven custom dictates that individuals keep
up or be left behind. Elves prefer to lead short, happy
lives rather than long, boring ones. Seeing the future as a
dark, deadly place, they prefer to live in "the now,"
enjoying each fleeting moment. They thrive in open
spaces, and tend to wither in captivity.\\
}
{Special Ability with } {\item \textbf{Named Option: } blaat} {} {} {} {} {}

\end{multicols*}

\end{document}
